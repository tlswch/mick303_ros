% presentation om cardis

\documentclass[aspectratio=169, beamer, oneside, leqno]{beamer}

\usepackage{presentation}

% titel, författare
\title{
  Presentation: Härledning av Levi-Civita-förbindelsen \\ % titel
  \\
  \large Ifrån antagande om ekvivalens mellan raka och stationära kurvor % undertitel
}
\author{Anders}
\date{\today\ (\currenttime)}

\begin{document}

% Titelsida
\maketitle

%% \tableofcontents

%% \clearpage

\setbeamertemplate{footline}[frame number]

\section{sample sec}

\begin{frame}
  \frametitle{\secname}
  \begin{itemize}[label=$\star$]
  \item Presentation ämnad för HW-gäng med kunskaper därefter
  \item Skrivet på min fritid, inte färdigt på långa vägar
  \item Mer intresserad? Kolla mina anteckningar och dess referenser!
  \item Syfte (och mål med denna presentation såklart)
    \begin{itemize}[label=$\wp$]
    \item Fördjupad teroetisk förståelse
    \item Specialfall
    \item Grundläggande implementation
    \item Attacker mot maskering
    \end{itemize}
  \end{itemize}
\end{frame}




\end{document}
