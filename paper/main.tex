% Latex-mall, gafflad från Bromskloss

%% =============================================================================
%% === PREAMBLE ================================================================
%% =============================================================================

% Dokumentklass
\documentclass[a4paper, article, oneside, leqno]{memoir}

% Ladda sty-fil preamble.sty
\usepackage{preamble}

% titel, författare, datum
\title{
  huvudtitel \\ % titel
  \large en undertitel% undertitel
}
\author{Anders}
\date{\today\ (\currenttime)}

\hypersetup{
  unicode,
  pdftitle={\thetitle},
  pdfauthor={\theauthor}
}

% Lägg till fil för referenser
% \addbibresource{referenser.bib}

\begin{document}

% Romerska siffror i förbladen
\pagenumbering{roman}

% Titelsida
\maketitle

% Sammanfattning
\begin{abstract}
  Häri beskrivs...
\end{abstract}

\clearpage

\tableofcontents

%% =============================================================================
%% === END PREAMBLE ============================================================
%% =============================================================================

\clearpage

% Siffror för artikeln själv
\pagenumbering{arabic}

%% =============================================================================
%% === BODY ====================================================================
%% =============================================================================

\chapter{Mängder}
Den fundamentala byggstenen är \emph{mängden}. En mängd kan vara tom eller innehålla element, som även de är mängder, sådana att detsamma gäller för dem. Den mängd som inte innehåller några element kallas för den \emph{tomma mängden} och är unik. Hädanefter är varje objekt en mängd, vilket inkluderar, men inte begränsas till, tal, vektorer, operatorer, funktioner och tensorer.

En mängd, $M$ med element $a$, $b$, $c$ kan skrivas som $M = \{a, b, c\}$, medan en mängd, $N$, med elementen $a$, $c$, $M$ och $d$, kan skrivas som $N = \{a, c, M, d\} = \{a, c, \{a, b, c\}, d\}$ \&sv. Den tomma mängden skrivs med denna notation $\{\}$ men andra symboler används, som $\varnothing$.

%% =============================================================================
%% === END BODY ================================================================
%% =============================================================================

\clearpage

%% =============================================================================
%% === APPENDICES ==============================================================
%% =============================================================================

\appendix

\chapter{Notation och definitioner} \label{sec:notation}
Notationen som används skiljer sig en del från standardnotation inom ingenjörsvärlden och drar mer åt det hållet som används inom teoretisk såväl som matetmatisk fysik. Notationen följer \cite{anders_mat_intro} som läsaren bör konsumerat vid det här laget.

$\Binary$ används för att beteckna mängden $\{0, 1\}$ som är mycket vanlig inom digital elektronik.

$\oplus$ används för den exklusiva eller-operatorn, som definieras bitvis enligt

\begin{align}\begin{split}
    \oplus : \Binary \times \Binary &\to \Binary \\
    (\alpha, \beta) &\mapsto \alpha \oplus \beta \\
    &= |\alpha - \beta|
\end{split}\end{align}

Och från den definitionen för längre bitsträngar

\begin{align}\begin{split}
    \oplus : \Binary^d \times \Binary^d &\to \Binary^d \\
    (\alpha_\mu, \beta_\mu) &\mapsto \alpha_\mu \oplus \beta_\mu \\
    &= |\alpha_\mu - \beta_\mu|_\mu
\end{split}\end{align}

Resultatet blir alltså en bitsträng i vilken varje komponent är resultaten av en $\oplus$ mellan motsvarande komponenter i operanderna.

% lägg in kod, include kan användas men den är högnivå
%\input{../slumpgenerator.tex}

%% =============================================================================
%% === END APPENDICES ==========================================================
%% =============================================================================

%% =============================================================================
%% === REFERENSER ==============================================================
%% =============================================================================

% skriv ut referenserna
%\printbibliography[title=Referenser]

%% =============================================================================
%% === END REFERENSER ==========================================================
%% =============================================================================

\end{document}
